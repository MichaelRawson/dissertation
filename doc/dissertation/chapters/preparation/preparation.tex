\chapter{Preparation}
After identifying the main goals of the project, these coarse requirements have to be refined, in order to be more precise about my aims and drive the project development.
The result of this is a set of \emph{requirements} that can then be analysed to predict problems and measure success.
Once any problems have been resolved, work can begin on implementation.

\section{Initial Choices}
Before the project could begin, several choices were made to make the project plan more concrete and define the problem space more accurately.
The first choice presented was choosing a proof assistant for use encoding the calculus.
My supervisor suggested Isabelle\cite{isabelle}, a proof assistant with strong links to the university.

The other choice that I made is the exact type system I would implement for my theory.
I chose to implement the simply-typed calculus (arguably the simplest), leaving any extensions until later in the project to allow for delays.

\section{Background Theory}
Some mathematical theory is required to state the problem more precisely, and can be used to direct the refinement of requirements.

\subsection{\(\lambda\)-Calculus}
The \(\lambda\)-calculus\cite{lambda-overview} is a formal system of computation represented by operations on an set of terms.
\begin{definition}
Terms \(M\) are inductively defined as follows.
\begin{enumerate}
\item
A variable, \(x\), is always a term.
These may be sub-categorised to be \emph{bound} if some \emph{binder} in an expression binds them, or \emph{free}, if there is no such binder.
\item
If \(M\) is a term, abstractions \(\lambda x.M\) are also terms.
This represents an anonymous function from terms to terms in the calculus (i.e. taking an argument \(x\) and returning \(M(x)\)), and therefore \emph{binds} \(x\) in \(M\).
\item
If \(M\) and \(N\) are terms, applying \(M\) to \(N\) is also a term, \(\wrap{M\ N}\).
This represents function application in the calculus, the dual of function abstraction.
\end{enumerate}
\end{definition}

This is relatively easy to define programmatically, as it can be represented with an algebraic datatype.
For instance, in Standard ML:
\begin{minted}{SML}
datatype 'a trm =
    Var of 'a
  | Fn  of ('a * 'a trm)
  | App of ('a trm * 'a trm)
\end{minted}

\subsection{The Problem of \(\alpha\)-Equivalence}
Unfortunately, this representation with names and binders has a problem: frequently, it is convenient to reason that e.g. \(\lambda x.x\) and \(\lambda y.y\) are the same: they do, after all, compute equal values on all inputs.

However, structurally-speaking they are not equal: the string \(x\) is not the same as \(y\).
This notion of two terms behaving the same regardless of using different names is \emph{\(\alpha\)-equivalence}.
\begin{definition}
The \(\alpha\)-equivalence relation \(\equiv_\alpha\) is the least congruence on terms such that
\[
\lambda x.M \equiv_\alpha \lambda y.M'
\]
where y does not occur free in \(M\), and \(M'\) is \(M\) with \(x\) substituted for \(y\) in a capture-avoiding fashion.
\end{definition}

Such a definition can be implemented in an assistant, then used whenever a statement of equivalence is required.
It also conveniently forms an equivalence relation.
It is nonetheless rather un-ergonomic and inefficient to carry such an assumption around, and assistants often reason better about actual equality than arbitrary equivalence relations.

\subsubsection{Quotient Types}
This problem can be solved by using \emph{quotient types}.
\begin{definition}
A quotient type \(Q\) consists of a base type \(R\), an equivalence relation \(\sim\) on \(R\), and functions \(\mathrm{Abs} : R \to Q\) and \(\mathrm{Rep} : Q \to R\).
Items \(q_1 : Q\) and \(q_2 : Q\) are equal iff \(Rep\ q_1 \sim Rep\ q_2\).
\end{definition}
Isabelle provides this mechanism via a \texttt{quotient\_datatype} command\cite{isabelle-quotient}, which takes \(R\) and \(\sim\) and produces \(Q\), \(\mathrm{Abs}\) and \(\mathrm{Rep}\).
It also provides ``lifting'' and ``transfer'' operations, useful for lifting operations from the base type to the quotient type, and transferring proof obligations from the quotient type to the base type.

This allows the definition of a type for terms modulo \(\alpha\)-equivalence as required by defining a datatype for pre-terms without a notion of equivalence (as before), then producing the new type as a quotient type over the \(\alpha\)-equivalence relation.

\subsection{Other Approaches to Names}
While we can now use equality directly instead of an equivalence relation, the definition of the equivalence relation is quite awkward to use.
There are several alternative ways of handling names and \(\alpha\)-equivalence, the most prominent being deBruijn indices, Higher-Order Abstract Syntax (or HOAS for short), and nominal techniques.

DeBruijn indices remove names altogether and instead uses natural numbers for bound variables to refer to the number of binders between the variable and the binder that bound it.
For instance, the constant function
\[
\lambda x. \lambda y. x
\]
becomes
\[
\lambda.\lambda. 1
\]
using deBruijn indices.
Using this approach, \(\alpha\)-equivalence is now simply equality, as all bound names have been removed.
The downside of this approach is that actually using this representation for proofs is quite difficult and unintuitive, and it conflicts with the general approach used in informal proof.

Higher-Order Abstract Syntax uses the environment's (in this case Isabelle's) own binder implementation (i.e. what it uses internally for its own function objects) to handle binding.
The datatype defined earlier for the calculus would then become
\begin{minted}{SML}
datatype 'a trm =
    Var of 'a
  | Fn  of ('a -> 'a trm)
  | App of ('a trm * 'a trm)
\end{minted}
and e.g. the constant function would be represented as \mintinline{SML}{Fn (fn x => Fn (fn y => x))}.
While this implementation neatly avoids many of the implementation issues of other approaches, it is not always possible to show certain properties of names with this representation\cite{HOAS}.

Finally, the techniques under the heading of ``nominal methods'' are a relatively-new approach of dealing with names, which importantly retain the explicit representation of names, as in the na\"ive version above.
The technique uses a different definition of \(\alpha\)-equivalence based on \emph{swapping} (rather than substituting) names in a given expression\cite{nominal}.
This was the approach that I chose.

\subsubsection{Swappings}
\begin{definition}
A swapping \([x \swap y]\) is a pair of variables.
\end{definition}
\begin{definition}
The effect of a swapping on a term, \([x \swap y] \cdot M\) is defined as
\[
[x \swap y] \cdot M =
\begin{cases}
y & M = x\\
x & M = y\\
z & M = z, z \notin \{x, y\}\\
\lambda ([x \swap y] \cdot z). \wrap{[x \swap y] \cdot N} & M = \lambda z.N\\
\wrap{[x \swap y] \cdot A}\ \wrap{[x \swap y] \cdot B} & M = \wrap{A\ B}
\end{cases}
\]
\end{definition}
An equivalence \(\sim\) can be defined using only this operation, as shown in figure \ref{fig:nominal}.
\begin{figure}
\begin{mathpar}
\inferrule[var]
 { }
 {x \sim x}

\inferrule[app]
 {A \sim C \\ B \sim D}
 {\wrap{A\ B} \sim \wrap{C\ D}}

\inferrule[fn]
 {[z \swap x] \cdot M \sim [z \swap y] \cdot N \\ z \# M \\ z \# N}
 {\lambda x.M \sim \lambda y.N}
\end{mathpar}
\caption{an equivalence defined in terms of swappings}
\label{fig:nominal}
\end{figure}
It can further be shown\cite{nominal} that \(\sim\) is precisely the same as \(\equiv_\alpha\), and hence can be used in place of it.

Therefore my strategy for representing terms modulo \(\alpha\)-equivalence will be to develop a theory of swappings, then use it to argue that \(\sim\) is an equivalence relation, and finally produce a new type as a quotient of the concrete type with respect to \(\sim\).

\subsubsection{Freshness}
Looking at the \textsc{Fn} rule of figure \ref{fig:nominal}, the preconditions \(z \# M\) and \(z \# N\) merit explanation: \(x \# M\) is the statement ``x is \emph{fresh} for M''.

\begin{definition}
An element \(x\) is fresh in a set \(S\) iff \(x \notin S\).
Moreover, a variable \(x\) is fresh for a term \(M\) iff \(x\) is fresh for the free variables of \(M\).
\end{definition}

The idea of picking a fresh item for a given set is common in proofs about the nominal approach to binding.
As a result, I also need a verified implementation of freshness to argue neatly about binding in Isabelle.

\subsection{Simple Types}
The untyped calculus can be extended to include a type system without any effect on the theory of names it uses: a \emph{type} is simply added to each binder, so \(\lambda x.M\) becomes \(\lambda (x:T).M\), for an arbitrary \(T\).

\begin{definition}
Simple types \(\tau\) are either
\begin{enumerate}
\item
A (fixed) type variable, say \(\alpha\) or \(\beta\).
\item
An arrow type \(\tau \to \tau\) from one type to another.
\end{enumerate}
\end{definition}

Adding simple types to the binders of the untyped calculus produces the \emph{simply-typed} \(\lambda\)-calculus.
The typing relation \(\Gamma \vdash M : \tau\) is given inductively in figure \ref{fig:typing}.
\(\Gamma\) here is a typing relation: a partial function from variables to types.

\begin{figure}
\begin{mathpar}
\inferrule[var]
 {\Gamma(x) = \tau}
 {\Gamma \vdash x : \tau}

\inferrule[fn]
 {\Gamma\{x \mapsto \tau\} \vdash M : \sigma}
 {\Gamma \vdash \lambda (x : \tau). M : \tau \to \sigma}

\inferrule[app]
 {\Gamma \vdash M : \tau \to \sigma \\ \Gamma \vdash N : \tau}
 {\Gamma \vdash \wrap{M\ N} : \sigma}
\end{mathpar}
\caption{typing rules for the simply-typed calculus}
\label{fig:typing}
\end{figure}

One advantage of the simply-typed calculus is that type inference is an extremely  simple algorithm, with no unification steps or other complexity that comes with more advanced type schemes.
A type inference algorithm \(\infertype(\Gamma, M)\) can be described recursively by
\[
\infertype\wrap{\Gamma, M} =
\begin{cases}
\Gamma(x) & M = x\\
\tau \to \infertype\wrap{\Gamma\{x \mapsto \tau\}, N} & M = \lambda (x : \tau). N\\
\mathrm{apply}\wrap{\infertype\wrap{\Gamma, A}, \infertype\wrap{\Gamma, B}} & M = \wrap{A\ B}
\end{cases}
\]
where \(\mathrm{apply}\wrap{\tau \to \sigma, \tau}\) produces \(\sigma\) for any \(\tau, \sigma\), and all other input is undefined.

Note that this algorithm produces the same type for all terms in an \(\alpha\)-equivalence class, and hence can be lifted to work on equivalence classes, rather than concrete terms.
\section{Requirements}
Moving from coarse requirements to finer ones is now more straightforward, as a number of constraints have been imposed by the mathematics.
Implementation in Isabelle can proceed as follows:
\begin{itemize}
\item
Develop formalised work about freshness and swappings to support later developments.
\item
Define a datatype for representing simple types.
\item
Define the datatype of \emph{pre-terms}, the type of terms before the \(\alpha\)-equivalence quotient.
\item
Define the effect of swappings on pre-terms.
\item
Define the \(\alpha\)-equivalence relation
\item
Prove sufficent lemmas to show it is an equivalence relation.
\item
Define a type inference algorithm on pre-terms and show that it is invariant under \(\alpha\)-equivalence.
\item
Produce a quotient type of \emph{terms} from the pre-terms and the equivalence relation.
\item
Lift any required definitions over the quotient, along with the type inference algorithm.
\item
Prove any required lemmas by reference to the same lemmas upon pre-terms.
\item
Define a typing relation inductively on the terms.
\item
Argue properties of this typing relation, such as the subject reduction property.
\item
Show that the inference algorithm is sound and complete with respect to the typing relation.
\item
Conclude that the implementation is verified and extract code.
\end{itemize}

\section{Starting Point}
As stated in my original project proposal, I was familiar with the majority of the theory required for implementing the \(\lambda\)-calculus and type theory in the project.
I was not, however, familiar with the number of neat approaches to name binding described above, or with Isabelle itself.
I am pleased to say that I have now learned a lot more about both of these topics.
