\chapter{Conclusion}
I have shown the development of a formalised implementation of a typed \(\lambda\)-calculus in the proof assistant Isabelle, complete with correctness properties about the type system and verified, extracted, code for type inference.
All success criteria have been met, and some extensions have been made, augmenting the core calculus and showing the confluence property.
This work differs from, and improves upon a typical implementation in its use of nominal techniques that have several advantages over other methods of name binding.

\section{List of Results}
Taking success criteria from my project proposal, they have all been met:
\begin{enumerate}
\item
I have now learned sufficient theory to understand, implement, and justify my approach to the problem.
\item
I gained sufficient practical experience before and during my project about the Isabelle proof assistant to efficiently implement the project.
\item
The representation of the calculus I chose has been sufficient to produce the rest of my dissertation with.
\item
I have proven the progress, preservation, and safety properties of the type system.
\item
The implementation of type inference has been verified by showing it equivalent to the inductive typing rules.
\item
The extracted Standard ML code does compile and run as expected.
Although Haskell was the language I eventually used for testing, I don't consider that this change of decision disqualifies this success criterion.
\item
The dissertation is complete.
\end{enumerate}

\section{Further Work}
There is significant scope for further work in this area.
One of several areas could be pursued:
\begin{itemize}
\item
Improving the nominal approach, perhaps adding some automation to remove some of the painful points.
\item
Extending or modifying the calculus to more interesting calculi, like System F.
\item
Improving performance of the extracted code.
\item
Further properties of the calculus, such as the strong normalisation property.
\end{itemize}
