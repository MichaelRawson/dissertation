\documentclass[12pt]{article}
\title{Part II Project Proposal: Formalising Simply-Typed \(\lambda\)-Calculus}
\author{Michael Rawson}

\begin{document}
\begin{titlepage}
%\thispagestyle{empty}

\rightline{\large Michael Rawson}
\medskip
\rightline{\large Robinson College}
\medskip
\rightline{\large mr644}

\vfil

\centerline{\large Part II Project Proposal, Computer Science Tripos}
\vspace{0.4in}
\centerline{\Large\bf Verified Metatheory and Type Inference}
\centerline{\Large\bf for a Name-Carrying Simply-Typed \(\lambda\)-Calculus}
\vspace{0.3in}
\centerline{\large \today}

\vfil

{\bf Project Originator:} Dr.~Dominic Mulligan

\vspace{0.1in}

{\bf Resources Required:} None

\vspace{0.5in}

{\bf Project Supervisor:} Dr.~Dominic Mulligan

\vspace{0.2in}

{\bf Signature:}

\vspace{0.5in}

{\bf Director of Studies:} Dr.~Alastair Beresford

\vspace{0.2in}

{\bf Signature:}

\vspace{0.5in}

{\bf Overseers:} Dr.~Ian Wassell and Prof.~Lawrence Paulson

\vspace{0.2in}

{\bf Signatures:}

\vfil
\end{titlepage}

\section*{Schedule}
The project is ahead of schedule by several weeks: to be more specific, the task assigned for the period of time 6/2--20/2 (verifying the type inference algorithm, and completing other remaining tasks) is nearing completion.

\section*{Unexpected Difficulties}
No \emph{completely} unexpected difficulties have arisen.
Some parts of the project have turned out to be more challenging or fiddly than others, notably in the treatment of names in the calculus.
However, dedicating sufficient time to them resolved all problems so far.

\section*{Progress}
My project involves representing, and proving some theorems about, the simply-typed $\lambda$-calculus.
I therefore started by choosing a representation for the calculus and its types within the Isabelle proof assistant.
Before proceeding with the rest of the project, an $\alpha$-equivalence relation was needed to avoid unnecessary details creeping into subsequent work: this was achieved by means of a representation with explicit names, combined with some work with \emph{permutations} of names in the representation, to establish an equivalence relation.

Once terms of the calculus were established, modulo $\alpha$-equivalence, the main part of the work could proceed.
I implemented typing and $\beta$-reduction relations on terms and types, then used these to show some key properties, such as the \emph{progress} and \emph{type-preservation} properties.
Finally, I have recently completed a type-inference algorithm for the calculus and provided soundness and completeness proofs of it, with regards to the inductively-defined typing relation.

In the remaining time, I hope to extend my project with (at least) some extra properties, such as \emph{confluence}.
Then, I will move on to the evaluation section of my project, which will likely involve testing and measuring aspects of the type inference algorithm, combined with comparisons of the proof script required with my approach compared to other Isabelle implementations of calculi with binders.
\end{document}
