\documentclass[12pt]{article}
\title{Part II Project Proposal: Formalising Simply-Typed \(\lambda\)-Calculus}
\author{Michael Rawson}

\begin{document}
\begin{titlepage}
%\thispagestyle{empty}

\rightline{\large Michael Rawson}
\medskip
\rightline{\large Robinson College}
\medskip
\rightline{\large mr644}

\vfil

\centerline{\large Part II Project Proposal, Computer Science Tripos}
\vspace{0.4in}
\centerline{\Large\bf Verified Metatheory and Type Inference}
\centerline{\Large\bf for a Name-Carrying Simply-Typed \(\lambda\)-Calculus}
\vspace{0.3in}
\centerline{\large \today}

\vfil

{\bf Project Originator:} Dr.~Dominic Mulligan

\vspace{0.1in}

{\bf Resources Required:} None

\vspace{0.5in}

{\bf Project Supervisor:} Dr.~Dominic Mulligan

\vspace{0.2in}

{\bf Signature:}

\vspace{0.5in}

{\bf Director of Studies:} Dr.~Alastair Beresford

\vspace{0.2in}

{\bf Signature:}

\vspace{0.5in}

{\bf Overseers:} Dr.~Ian Wassell and Prof.~Lawrence Paulson

\vspace{0.2in}

{\bf Signatures:}

\vfil
\end{titlepage}

\section*{Schedule}
The project is ahead of schedule by several weeks: to be more specific, the task assigned for the period of time 6/2--20/2 is nearing completion.

\section*{Unexpected Difficulties}
No \emph{completely} unexpected difficulties have arisen.
Some parts of the project have turned out to be more challenging or fiddly than others, notably in the treatment of names in the calculus.
However, dedicating sufficient time to them resolved all problems so far.

\section*{Progress}
So far, the following elements of my project are complete:
\begin{itemize}
\item
datatypes for representing the the terms of the simply-typed $\lambda$-calculus and their types
\item
theory about permutations and their composition, application, and application to sets
\item
theory applying permutations to the simply-typed calculus
\item
an $\alpha$-equivalence relation defined using permutations, and a proof showing the relation is an equivalence relation
\item
a quotient type defining $\lambda$-terms modulo $\alpha$-equivalence
\item
``lifting'' of definitions and lemmas defined on the raw terms to the quotiented terms
\item
an inductively-defined typing relation, and a $\beta$-reduction relation
\item
proofs of several properties about the calculus, using these relations --- including progress and type-preservation
\item
a type inference algorithm defined on quotiented terms
\item
proofs of the algorithm's soundness and completeness
\end{itemize}
\end{document}
